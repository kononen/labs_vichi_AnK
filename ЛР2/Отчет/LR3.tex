 \documentclass[12pt,a4paper]{article}

\usepackage[utf8]{inputenc}
\usepackage[T2A]{fontenc}
\usepackage[russian]{babel}
\usepackage{hyperref}
%\usepackage[footnotes,oglav,spisok,boldsect,eqwhole,remarks,kursrab, hyperprint]{project}
\newcommand{\cond}{\mathop{\mathrm{cond}}}

\begin{document}
\cover{Методы вычислений}{Итерационные методы решения систем линейных алгебраических уравнений}{студенты группы ФН2-51}{Бондарчук~Виктория и\: Мукова~Рагнеда}{\centerline{\Large{Вариант 2; 7}}}{}{}{}{2016}
%\tableofcontents
\newpage

\section{Исходные данные}

Рассмотрим систему линейных алгебраических уравнений (СЛАУ) вида:
\begin{gather}
A x=b,
\label{sys}
\end{gather}
где $A$ --- матрица системы размера $n \times n$, $\det A \ne 0$, $b$ --- вектор правой части.

\section{Описание используемых алгоритмов}

Систему (\ref{sys}) можно привести к виду:
\begin{gather}
x=C x+y,
\label{sys2}
\end{gather}
где $C$ --- квадратная матрица $n \times n$, $y$ --- вектор-столбец. Это можно сделать различными способами. Формула (\ref{sys2}) подсказывает рекурентное соотношение вида:
\begin{gather}
x^{k+1}=C x^k+y, \qquad k=0, 1, \ldots.
\label{rek}
\end{gather}
Достаточным условием сходимости итерационного процесса является выполнение условия:
$$
\|C\|\le 1.
$$

\subsection{Метод простой итерации}
Методом простой итерации называется явный метод
\begin{gather}
\frac{x^{k+1}-x^k}{\tau}+A x^k=b, \quad k=0,1,\ldots.
\end{gather}
Здесь $\tau$ --- это специально выбираемый итерационный параметр.

Для метода простой итерации систему можно записать в виде:
\begin{gather}
x=-(\tau A-E) x+\tau b.
\end{gather}
Из сравнения с (\ref{rek}) получаем, что для метода простой итерации матрица $C$ и вектор $y$ выражаются в виде:
$$
C=-(\tau A-E), \qquad y=\tau b.
$$ 

\subsection{Метод Якоби}

Представим матрицу $A$ в виде суммы
$$
A=D+L+U,
$$
где $L$ --- нижняя треугольная матрица, $U$ --- верхняя треугольная матрица, $D=\text{diag}\{a_{11},a_{22},\ldots,a_{nn}\}$ --- диагональ матрицы.

Из первого уравнения системы (\ref{sys}) выразим переменную $x_1$:
$$
x_1=\frac{(b_1-a_{12} x_2-a_{13} x_3-\ldots-a_{1n} x_n)}{a_{11}}.
$$
Из второго уравнения --- переменную $x_2$:
$$
x_2=\frac{(b_2-a_{21} x_1-a_{23} x_3-\ldots-a_{2n} x_n)}{a_{22}}
$$
и т.д. В результате получим систему вида
$$
x=C x+y,
$$
или в покомпонентной записи
\begin{gather*}
x_1=\quad c_{12} x_2+c_{13} x_3+\ldots+c_{1n} x_n+y_1\\
x_2=c_{21} x_1+ \quad c_{23} x_3+\ldots+c_{2n} x_n+y_2\\
\ldots\\
x_n=c_{n1} x_1+c_{n2} x_2+c_{n3} x_3+\ldots+y_n.  
\end{gather*}

На главной диагонали матрицы $C$ находятся нулевые элементы, остальные коэффициенты матрицы и компоненты вектора $y$ вычисляются по формулам
\begin{gather*}
c_{ij}=-\frac{a_{ij}}{a_{ii}}, \quad y_i=\frac{b_i}{a_{ii}}, \quad i,j=1,\ldots,n, \, i \ne j.
\end{gather*}
Такой выбор матрицы $C$ и вектора $y$ соответствует методу Якоби.
$$
C=-D^{-1} (L+U), \quad y=D^{-1} b.
$$

\subsection{Методы Зейделя и релаксации}

Каноническая форма метода Зейделя имеет вид 
$$
(D+L) (x^{k+1}-x^k)+A x^k=b, \quad k=0,1,2,\ldots.
$$

Метод Зейделя является частным случаем метода релаксации, задаваемого в виде
\begin{gather}
(D+\omega L) \frac{x^{k+1}-x^k}{\omega}+A x^k=b, \quad k=0,1,2,\ldots.
\label{mR}
\end{gather}
Здесь $\omega>0$ --- заданный числовой параметр --- параметр релаксации.

Чтобы привести систему (\ref{mR}) к виду, удобному для итераций, перепишем ее в виде
$$
(E+\omega D^{-1} L) x^{k+1}=((1-\omega) E-\omega D^{-1} U)x^k+\omega D^{-1} b.
$$
В покомпонентной записи
$$
x^{k+1}_i+\omega \sum\limits^{i-1}_{j=1} \frac{a_{ij}}{a_ii} x^{k+1}_j=(1-\omega) x_i^k-\omega \sum\limits_{j=i+1}^n\frac{a_{ij}}{a_{ii}}x_j^k+\omega \frac{b_i}{a_{ii}}, \, i=1,2,\ldots,n.
$$
Отсюда можно последовательно найти все $x_i^{k+1}$, $i=1,2,\ldots,n$.

\subsection{Критерий останова итерационного процесса}

Критерием прекращения итераций, обеспечивающим достижение заданной точности $\varepsilon$, является выполнение неравенства
$$
\|x^{k+1}-x^k\| \le \frac{1-\|C\|}{\|C\|}\varepsilon.
$$

Для метода Зейделя существует так же другой критерий:
$$
\|x^{k+1}-x^k\| \le \frac{1-\|C\|}{\|C_U\|}\varepsilon.
$$

\section{Результаты рассчетов}
\figbox{15cm}{19cm}{11}{obtosh}[<<Тест 1>>. Расчет с $\varepsilon=10^{-4}$.]\\
\\
\figbox{15cm}{19cm}{12}{obtosh}[<<Тест 1>>. Расчет с $\varepsilon=10^{-4}$.]\\
\\
\figbox{15cm}{19cm}{13}{obtosh}[<<Тест 1>>. Расчет с $\varepsilon=10^{-4}$.]\\
\\
\figbox{15cm}{5cm}{14}{obtosh}[<<Тест 1>>. Расчет с $\varepsilon=10^{-4}$.]\\
\\
\figbox{15cm}{19cm}{21}{obtosh}[<<Тест 1>>. Расчет с $\varepsilon=10^{-7}$.]\\
\\
\figbox{15cm}{19cm}{22}{obtosh}[<<Тест 1>>. Расчет с $\varepsilon=10^{-7}$.]\\
\\
\figbox{15cm}{19cm}{23}{obtosh}[<<Тест 1>>. Расчет с $\varepsilon=10^{-7}$.]\\
\\
\figbox{15cm}{5cm}{24}{obtosh}[<<Тест 1>>. Расчет с $\varepsilon=10^{-7}$.]\\
\\
\figbox{15cm}{19cm}{31}{obtosh}[<<Тест 2>>. Расчет с $\varepsilon=10^{-4}$.]\\
\\
\figbox{15cm}{19cm}{32}{obtosh}[<<Тест 2>>. Расчет с $\varepsilon=10^{-4}$.]\\
\\
\figbox{15cm}{19cm}{33}{obtosh}[<<Тест 2>>. Расчет с $\varepsilon=10^{-4}$.]\\
\\
\figbox{15cm}{5cm}{34}{obtosh}[<<Тест 2>>. Расчет с $\varepsilon=10^{-4}$.]\\
\\
\figbox{15cm}{19cm}{41}{obtosh}[<<Тест 2>>. Расчет с $\varepsilon=10^{-7}$.]\\
\\
\figbox{15cm}{19cm}{42}{obtosh}[<<Тест 2>>. Расчет с $\varepsilon=10^{-7}$.]\\
\\
\figbox{15cm}{19cm}{43}{obtosh}[<<Тест 2>>. Расчет с $\varepsilon=10^{-7}$.]\\
\\
\figbox{15cm}{5cm}{44}{obtosh}[<<Тест 2>>. Расчет с $\varepsilon=10^{-7}$.]\\
\\
\figbox{15cm}{19cm}{51}{obtosh}[<<Вариант 2>>. Расчет с $\varepsilon=10^{-4}$.]\\
\\
\figbox{15cm}{19cm}{52}{obtosh}[<<Вариант 2>>. Расчет с $\varepsilon=10^{-4}$.]\\
\\
\figbox{15cm}{19cm}{53}{obtosh}[<<Вариант 2>>. Расчет с $\varepsilon=10^{-4}$.]\\
\\
\figbox{15cm}{5cm}{54}{obtosh}[<<Вариант 2>>. Расчет с $\varepsilon=10^{-4}$.]\\
\\
\figbox{15cm}{19cm}{61}{obtosh}[<<Вариант 2>>. Расчет с $\varepsilon=10^{-7}$.]\\
\\
\figbox{15cm}{19cm}{62}{obtosh}[<<Вариант 2>>. Расчет с $\varepsilon=10^{-7}$.]\\
\\
\figbox{15cm}{19cm}{63}{obtosh}[<<Вариант 2>>. Расчет с $\varepsilon=10^{-7}$.]\\
\\
\figbox{15cm}{5cm}{64}{obtosh}[<<Вариант 2>>. Расчет с $\varepsilon=10^{-7}$.]\\
\\
\figbox{15cm}{19cm}{71}{obtosh}[<<Вариант 7>>. Расчет с $\varepsilon=10^{-4}$.]\\
\\
\figbox{15cm}{19cm}{72}{obtosh}[<<Вариант 7>>. Расчет с $\varepsilon=10^{-4}$.]\\
\\
\figbox{15cm}{19cm}{73}{obtosh}[<<Вариант 7>>. Расчет с $\varepsilon=10^{-4}$.]\\
\\
\figbox{15cm}{5cm}{74}{obtosh}[<<Вариант 7>>. Расчет с $\varepsilon=10^{-4}$.]\\
\\
\figbox{15cm}{19cm}{81}{obtosh}[<<Вариант 7>>. Расчет с $\varepsilon=10^{-7}$.]\\
\\
\figbox{15cm}{19cm}{82}{obtosh}[<<Вариант 7>>. Расчет с $\varepsilon=10^{-7}$.]\\
\\
\figbox{15cm}{19cm}{83}{obtosh}[<<Вариант 7>>. Расчет с $\varepsilon=10^{-7}$.]\\
\\
\figbox{15cm}{5cm}{84}{obtosh}[<<Вариант 7>>. Расчет с $\varepsilon=10^{-7}$.]\\
\\
\figbox{15cm}{5cm}{BM1}{obtosh}[<<Большая матрица>>. Расчет с $\varepsilon=10^{-4}$.]\\
\\
\figbox{15cm}{5cm}{BM2}{obtosh}[<<Большая матрица>>. Расчет с $\varepsilon=10^{-7}$.]\\

\section{Анализ результатов}

В лабораторной работе решаются системы линейных алгебраических уравнений с точностями $\varepsilon=10^{-4}$ и $\varepsilon=10^{-7}$. При решении системы с $\varepsilon=10^{-7}$ вектор решения $x$ мы получаем с точностью до седьмого знака после запятой (при критерии останова $\|x^{k+1}-x^k\|<\cfrac{1-\|C\|}{\|C\|} \varepsilon$).

При решении СЛАУ предложенными итерационными методами, можно сделать вывод, что метод простой итерации сходится медленнее, чем остальные. Методы Зейделя и релаксации сходятся быстрее, чем методы простой итерации и метод Якоби. В пакете Wolfram Mathematica была проведена оценка количества операций: в некоторых случаях теоретическая оценка была близка к реально выполненному числу итераций, в некоторых она получилась намного меньше.

\section{Ответы на контрольные вопросы}
\begin{enumerate}
\item \textbf{Почему условие $\|C\|<1$ гарантирует сходимость итерационных методов?}\\
\\
Стационарный итерационнный метод:
\begin{gather}
x^{(k+1)}=C x^{(k)}+f
\label{eqat}
\end{gather}
Так как $\|C\|<1$, то оператор $C$ является сжимающим, то есть уравнение (\ref{eqat}) имеет одно решение (существует единственная неподвижная точка $x$, к которой <<стягивается>> отображение; эта точка является решением СЛАУ). Следовательно, метод сходится.
\\
\item \textbf{Каким следует выбирать итерационный параметр $\tau$ в методе простой итерации для увеличения скорости сходимости? Как выбрать начальное приближение $x^0$?}\\
\\
\par
Параметр $\tau$ выбирают так, чтобы по возможности сделать минимальной величину $\|C\|$. \\
Теорема Самарского: Пусть $A$ --- симметричная положительно определенная матрица, $B$ --- положительно определенная матрица. Тогда для того чтобы итерационная последовательность, определяемая соотношением 
$$
B \frac{x^{(k+1)}-x^{(k)}}{\tau}+A x^{(k)}=b
$$
при любом выборе нулевого приближения $x^0$ сходилась к точному решению $x$ системы $A x=b$, достаточно, чтобы были выполнены условия:
\begin{gather}
2 B>\tau A, \, \tau A>0.
\label{usl}
\end{gather}
Если матрица $B$ является симметричной, то условия (\ref{usl}) не только достаточны, но и необходимы для сходимости указанной итерационной последовательности при любом выборе нулевого приближения $x^0$.
\par
Оптимальное значение параметра $\tau_{opt}=\dfrac{2}{|\lambda_{max}|+|\lambda_{min}|}$, где $\lambda_{max}, \lambda_{min} \in \mathbb{C}$ --- максимальное и минимальное собственные значения матрицы $A$.\\
Следовательно, если матрица $A$ --- симметричная и $A>0, B>0$, то параметр $\tau$ выбирается из условия $\tau \le \dfrac{2}{|\lambda_{max}|}$ или $\tau_{opt}=\dfrac{2}{|\lambda_{max}|+|\lambda_{min}|}$.\\
\par
Если матрицы $A, B$ не удовлетворяют условиям теоремы, то параметр $\tau$ следует выбирать из условия $\tau<\dfrac{2}{|\lambda_{max}|}$. Оптимальное значение параметра $\tau$ выбирается из условия $\tau_{opt}<\dfrac{2}{|\lambda_{max}|+|\lambda_{min}|}$.\\
По условию теоремы, начальное приближение может быть любым, следовательно можно взять за начально приближение вектор правой части $x^0=b$. 
\\
\item \textbf{На примере систем из двух уравнений с двумя неизвестными дайте геометрическую интерпретацию метода Якоби, метода Зейделя, метода релаксации.}\\
\\
Рассмотрим систему уравнений:
\begin{gather*}
\begin{cases}
a_{11} x_1+a_{12} x_2=b_1,\\
a_{21} x_1+a_{22} x_2=b_2.
\end{cases}
\end{gather*}
\begin{enumerate}
\item Метод Якоби:\\
\begin{gather*}
D=
\begin{pmatrix}
a_{11} & 0\\
0 & a_{22}
\end{pmatrix}, \quad \tau=1.\\
\begin{cases}
a_{11} x_1^{(k+1)}+a_{12} x_2^{(k)}=b_1,\\
a_{21} x_1^{(k)}+a_{22} x_2^{(k+1)}=b_2.
\end{cases}
\end{gather*}
\\
Считаем, что $x^{(k)}$ известно, тогда:
\begin{gather*}
\begin{cases}
x_1^{(k+1)}=\frac{1}{a_{11}} (b_1-a_{12} x_2^{(k)}),\\
x_2^{(k+1)}=\frac{1}{a_{22}} (b_2-a_{21} x_1^{(k)}).
\end{cases}
\end{gather*}
\item Метод Зейделя:\\
\begin{gather*}
D=
\begin{pmatrix}
a_{11} & 0\\
0 & a_{22}
\end{pmatrix}, \quad
L=
\begin{pmatrix}
0 & 0\\
a_{21} & 0
\end{pmatrix}, \quad \tau=1.\\
\begin{cases}
a_{11} x_1^{(k+1)}+a_{12} x_2^{(k)}=b_1,\\
a_{21} x_1^{(k+1)}+a_{22} x_2^{(k+1)}=b_2.
\end{cases}
\end{gather*}
\\
Считаем, что $x^{(k)}$ известно, тогда:
\begin{gather*}
\begin{cases}
x_1^{(k+1)}=\frac{1}{a_{11}} (b_1-a_{12} x_2^{(k)}),\\
x_2^{(k+1)}=\frac{1}{a_{22}} (b_2-a_{21} x_1^{(k+1)}).
\end{cases}
\end{gather*}
\end{enumerate}
\item Метод релаксации:\\
\begin{gather*}
D=
\begin{pmatrix}
a_{11} & 0\\
0 & a_{22}
\end{pmatrix}, \quad
L=
\begin{pmatrix}
0 & 0\\
a_{21} & 0
\end{pmatrix}, \quad 
U=
\begin{pmatrix}
0 & 0\\
0 & a_{22}
\end{pmatrix}, \quad \tau=\omega\in\mathbb{R}.\\
\\
\Biggl(\frac{D}{\omega}+L\Biggr)x^{(k+1)}+\Biggl(U+\frac{\omega-1}{\omega} D\Biggr) x^{(k)}=b \Longrightarrow
\begin{cases}
x_1^{(k+1)}=\frac{\omega}{a_{11}} (b_1-a_{12} x_2^{(k)}-\frac{a_{11}(\omega-1)}{\omega} x_1^{(k)})\\
x_2^{(k+1)}=\frac{\omega}{a_{22}} (b_2-a_{21} x_1^{(k+1)}-\frac{a_{22} (\omega-1)}{\omega} x_2^{(k)})
\end{cases}
\end{gather*}
\\
Геометрическая интерпретация:\\
\figbox{11cm}{9cm}{MJ}{jacobi}[Метод Якоби.]\\
\\
\figbox{9cm}{9cm}{MZ}{zaidel}[Метод Зейделя.]\\
\\
\figbox{9cm}{9cm}{MSR}{relax}[Метод релаксации.]\\
\\
\item \textbf{При каких условиях сходятся метод простой итерации, метод Якоби, метод Зейделя и метод релаксации? Какую матрицу называют положительно определенной?}\\
\\
\begin{enumerate}
\item Метод простой итерации:\\
Если матрица $A$ симметричная и положительно определенная, то метод простой итерации сходится, если $\tau<\frac{2}{\lambda_{max}}$, где $\lambda_{max}$ --- максимальное собственное значение матрицы $A$.\\

Достаточным условием сходимости метода простой итерации является выполнение условия $\|C\|<1$.
\item Метод Якоби:\\
Пусть $A$ --- симметричная положительно определенная матрица. $a_{ii}>\sum\limits_{j \ne i} |a_{ij}| \, \forall i=\overline{1,n}$. Тогда метод Якоби сходится.
\item Метод релаксации:\\
Пусть матрица $A$ --- симметричная положительно определенная матрица. Тогда метод релаксации сходится при $0<\omega<2$.
\item Метод Зейделя:\\
Метод Зейделя является частным случаем метода релаксации ($\omega=1$). Для того, чтобы метод Зейделя сходился необходимо и достаточно, чтобы все корни уравнения $\det (U+\lambda L)=0$, где $L$ --- нижний треугольник матрицы $A$, $U$ --- верхний треугольник матрицы $A$, по модулю были меньше 1.\\
\par
Матрица $A$ называется положительно определенной, если $\forall x\in \mathbb{R}^n$  $\exists \delta >0:$ $(A x,x) \ge \delta ||x||^2$.
\end{enumerate}
\\
\item \textbf{Выпишите матрицу $C$ для методов Зейделя и релаксации.}\\
Для метода Зейделя матрица $C$ имеет вид:\\
 $C=-(D+L)^{-1} U$, где $D$ --- диагональ матрицы $A$, $L$ --- нижний треугольник матрицы $A$, $U$ --- верхний треугольник матрицы $A$.\\
Для метода релаксации матрица $C$ имеет вид:\\
$C=-\Biggr(\dfrac{D}{\omega}+L\Biggl)^{-1} \Biggl(U+\dfrac{\omega-1}{\omega} D\Biggr)$.
\\
\item \textbf{Почему в общем случае для остановки итерационного процесса нельзя использовать критерий $\|x^k-x^{k-1}\|<\varepsilon$?}\\
\\
В случаях, когда $\dfrac{1-\|C\|}{\|C\|}\ll 1$ использование критерия $\|x^k-x^{k-1}\|<\varepsilon$ приводит к существенному преждервременному окончанию итераций. Величина $\|x^k-x^{k-1}\|$ оказывается малой не потому, что приближения $x^{(k-1)}$ и $x^{(k)}$ близки, а потому что метод сходится медленно. 
\\
\item \textbf{Какие еще критерии окончания итерационного процесса Вы можете предложить?}\\
Можно воспользоваться условием для невязки:
\begin{gather*}
r_{k+1}=\|A x^{(k+1)}-f\| \le \varepsilon,
\end{gather*}
причем итерационный процесс сходится, если $\cfrac{r_{k+1}}{r_0}<\varepsilon$.\\
Критериями останова так же могут служить условия:\\
\begin{gather*}
\Bigr\|\frac{x^{k+1}-x^k}{|x^k|+\varepsilon_0}\Bigl\| \le \varepsilon\\
\|A x^{(k+1)}-f\|\le \varepsilon\\
\|x^{k+1}-x^k\|\le \varepsilon \|x^k\|+\varepsilon_0.
\end{gather*}
\end{enumerate}
\end{document} 